\graphicspath{./Images}
\section*{Appendix B: Objective constitutive relation}
\setcounter{table}{0}
\renewcommand{\thetable}{A\arabic{table}}
\renewcommand{\theequation}{A.\arabic{equation}}
\renewcommand{\thefigure}{A.\arabic{figure}}
\addcontentsline{toc}{section}{\protect\numberline{}Appendix B: Objective constitutive relation}%
\subsection*{B.1 Derivation of objective constitutive equation}
\addcontentsline{toc}{subsection}{\protect\numberline{}B.1: Derivation of objective constitutive equation}%
The derivation of the objective constitutive equation that is described in this appendix is based on Chapter 8 of Chaves \cite{chaves}.\\
\newline
To find a suitable constitutive relation in rate format, we start by the general form of linearised material behaviour as defined in the undeformed configuration, including the 2nd Piola-Kirchhoff stress $\boldsymbol{S}$, the material stiffness tensor $\boldsymbol{\mathbb C}$ and the Green-Lagrange strain $\boldsymbol{E}$. The general form is given by:
\begin{equation}
\boldsymbol{S} = \boldsymbol{\mathbb C}:\boldsymbol{E}.
\end{equation}
The 2nd Piola-Kirchhoff stress relates to the Cauchy stress tensor by:
\begin{equation}
\boldsymbol{S} = J\;\boldsymbol{F}^{-1}\cdot\boldsymbol{\sigma}\cdot\boldsymbol{F}^{-T}.
\end{equation}
In which $J$ is the volume ratio defined as $J=\text{det}(\boldsymbol{F})$.         Substituting this expression in the general form, and solving for the Cauchy stress tensor leads to:
\begin{equation}\label{AppB_cauchy}
\boldsymbol{\sigma} = \frac{1}{J}\;\boldsymbol{F}\cdot(\boldsymbol{\mathbb C}:\boldsymbol{E})\cdot\boldsymbol{F}^T.
\end{equation}
We take the time derivative now, considering that the material elasticity tensor $\boldsymbol{\mathbb C}$ is constant and using the chain-rule:
\begin{equation}\label{AppB_rateinterm1}
\begin{split}
\dot{\boldsymbol{\sigma}} = -\frac{\dot{J}}{J^2}\boldsymbol{F}\cdot(\boldsymbol{\mathbb C}:\boldsymbol{E})\cdot\boldsymbol{F}^T + \frac{1}{J}\;\dot{\boldsymbol{F}}\cdot(\boldsymbol{\mathbb C}:\boldsymbol{E})\cdot\boldsymbol{F}^T + \\
\frac{1}{J}\;\boldsymbol{F}\cdot(\boldsymbol{\mathbb C}:\dot{\boldsymbol{E}})\cdot\boldsymbol{F}^T+\frac{1}{J}\;\boldsymbol{F}\cdot(\boldsymbol{\mathbb C}:\boldsymbol{E})\cdot\dot{\boldsymbol{F}}^T
\end{split}
\end{equation}
We introduce some definitions:
\begin{equation}
\text{tr}(\boldsymbol{D}) = \frac{\dot{J}}{J}, \quad \dot{\boldsymbol{F}}=\boldsymbol{L}\cdot\boldsymbol{F},\quad \dot{\boldsymbol{F}}^T=\boldsymbol{F}^T\cdot\boldsymbol{L}^T,\quad \dot{\boldsymbol{E}}=\boldsymbol{F}^T\cdot\boldsymbol{D}\cdot\boldsymbol{F},
\end{equation}
in which the velocity gradient tensor $\boldsymbol{L}$ and the rate of deformation tensor $\boldsymbol{D}$ are given by
\begin{equation}
\boldsymbol{L}=\dot{\boldsymbol{F}}\cdot\boldsymbol{F}^{-1}\quad\text{and}\quad \boldsymbol{D}=\frac{1}{2}\Big(\boldsymbol{L}+\boldsymbol{L}^T\Big).
\end{equation}
Substituting the definitions into the expression for the time derivative of the Cauchy stress tensor of Equation \ref{AppB_rateinterm1} leads to:
\begin{equation}
\begin{split}
\dot{\boldsymbol{\sigma}} =& -\text{tr}(\boldsymbol{D})\frac{1}{J}\; \boldsymbol{F}\cdot(\boldsymbol{\mathbb C}:\boldsymbol{E})\cdot\boldsymbol{F}^T +\frac{1}{J}\;\boldsymbol{L}\cdot\boldsymbol{F}\cdot(\boldsymbol{\mathbb C}:\boldsymbol{E})\cdot\boldsymbol{F}^T +\\
&\frac{1}{J}\;\boldsymbol{F}\cdot\Big(\boldsymbol{\mathbb C}:(\boldsymbol{F}^T\cdot\boldsymbol{D}\cdot\boldsymbol{F})\Big)\cdot\boldsymbol{F}^T+\frac{1}{J}\;\boldsymbol{F}\cdot(\boldsymbol{\mathbb C}:\boldsymbol{E})\cdot\boldsymbol{F}^T\cdot\boldsymbol{L}^T.
\end{split}
\end{equation}
In this big equation, we recognize the expression for the Cauchy stress that was given in Equation \ref{AppB_cauchy}. Substituting this leads to:
\begin{equation}
\dot{\boldsymbol{\sigma}} = -\text{tr}(\boldsymbol{D})\boldsymbol{\sigma} +\boldsymbol{L}\cdot\boldsymbol{\sigma} + \frac{1}{J}\;\boldsymbol{F}\cdot\Big(\boldsymbol{\mathbb C}:(\boldsymbol{F}^T\cdot\boldsymbol{D}\cdot\boldsymbol{F})\Big)\cdot\boldsymbol{F}^T+\boldsymbol{\sigma}\cdot\boldsymbol{L}^T.
\end{equation}
This equation can be rearranged to:
\begin{equation}\label{AppB_rateinterm2}
\dot{\boldsymbol{\sigma}} - \boldsymbol{L}\cdot\boldsymbol{\sigma} - \boldsymbol{\sigma}\cdot\boldsymbol{L}^T = -\text{tr}(\boldsymbol{D})\boldsymbol{\sigma} + \frac{1}{J}\;\boldsymbol{F}\cdot\Big(\boldsymbol{\mathbb C}:(\boldsymbol{F}^T\cdot\boldsymbol{D}\cdot\boldsymbol{F})\Big)\cdot\boldsymbol{F}^T.
\end{equation}
On the left hand side of this equation, the expression for the Lie-derivative of the Cauchy stress can be recognized. Also, a part of the second term on the right hand side can be rewritten to match the expression for the spatial elasticity tensor $\boldsymbol{\mathbb L}$ that was given in Equation \ref{Ch3_CL}. 
\begin{equation}
\begin{split}
\frac{1}{J}\;\boldsymbol{F}\cdot\Big(\boldsymbol{\mathbb C}:(\boldsymbol{F}^T\cdot\boldsymbol{D}\cdot\boldsymbol{F})\Big)\cdot\boldsymbol{F}^T &= \frac{1}{J}\;(\boldsymbol{F}\overline{\otimes}\boldsymbol{F}):\boldsymbol{\mathbb C}:(\boldsymbol{F}^T\underline{\otimes}\boldsymbol{F}^T):\boldsymbol{D}, \\
&= \frac{1}{J}\; \boldsymbol{\mathbb L}:\boldsymbol{D}.
\end{split}
\end{equation}



Thus, Equation \ref{AppB_rateinterm2} can be rewritten as:
\begin{equation}
\acctriang{\boldsymbol{\sigma}} = -\text{tr}(\boldsymbol{D})\boldsymbol{\sigma} + \frac{1}{J}\; \boldsymbol{\mathbb L}:\boldsymbol{D}.
\end{equation}
Since the term with $\text{tr}(\boldsymbol{D})$ is negligibly small in most cases, this term will be neglected. With this assumption, we find a constitutive rate equation of the form:
\begin{equation}\label{AppB_consteq1}
\acctriang{\boldsymbol{\sigma}} = \frac{1}{J}\; \boldsymbol{\mathbb L}:\boldsymbol{D}.
\end{equation} 
Or, with plasticity included, it becomes:
\begin{equation}\label{AppB_consteq}
\acctriang{\boldsymbol{\sigma}} = \frac{1}{J}\; \boldsymbol{\mathbb L}:(\boldsymbol{D}-\boldsymbol{D}^p).
\end{equation} 

\newpage
\subsection*{B.2 Incremental stress integration}
\addcontentsline{toc}{subsection}{\protect\numberline{}B.2: Incremental stress integration}%
For application in a numerical code, an incremental integration of the constitutive rate equation of \ref{AppB_consteq} needs to be performed to obtain the Cauchy stress. To compose an incremental integration, we will make use of the mid-point rule as described in De Souza Neto et al.\cite{compmethodsplasticity}{\color{red}{check this reference}}.\\ 
\newline
We start with defining a discrete time interval $[t_n, t_{n+1}]$, with a length of $\Delta t$. The deformation gradients $\boldsymbol{F}_n$ and $\boldsymbol{F}_{n+1}$ describe the transformation from the undeformed configuration to the configurations at $t_n$ and $t_{n+1}$ respectively. At the beginning of the interval, the stress state is defined as $\boldsymbol{\sigma}_n$, and at the end of the interval the stress state is defined as $\boldsymbol{\sigma}_{n+1}$.\\
\newline
With the stress states $\boldsymbol{\sigma}_n$ and $\boldsymbol{\sigma}_{n+1}$ we can approximate the derivative of the Cauchy stress in the mid-point of the time interval, at $t_{n+\frac{1}{2}}$, as follows:
{\begin{equation}\label{AppB_incrderr1}
\dot{{\boldsymbol{\sigma}}}_{n+\frac{1}{2}}\approx\frac{{\boldsymbol{\sigma}}_{n+1}-{\boldsymbol{\sigma}}_{n}}{\Delta t}.
\end{equation}
However, we are using the Lie-derivative of the Cauchy stress tensor in stead of the regular time derivative. The Lie-derivative is obtained by the pull-back push-forward procedure as described in Appendix A. The first step of the pull-back procedure for the Lie-derivative is to transform the stress to the undeformed configuration. We define these transformed stresses at time $t_n$ and $t_{n+1}$ as:
\begin{equation}
\left.\begin{aligned}
\bar{\boldsymbol{\sigma}}_{n+1} &=\boldsymbol{F}_{n+1}^{-1}\cdot\boldsymbol{\sigma}_{n+1}\cdot\boldsymbol{F}_{n+1}^{-T}\\
\bar{\boldsymbol{\sigma}}_{n} &= \boldsymbol{F}^{-1}_{n}\cdot\boldsymbol{\sigma}_{n}\cdot\boldsymbol{F}_{n}^{-T},
\end{aligned}\right.
\label{Ch3_stressbar}
\end{equation}
with the bar above the symbol denoting that it has been transformed to the undeformed configuration.\\
\newline
The second step in the pull-back push-forward procedure is to take the time derivative in this transformed state. This can be approximated in a similar way as in Equation \ref{AppB_incrderr1}, only now with bars over the tensor symbols, denoting that the tensors live in the undeformed configuration.
{\begin{equation}
\dot{\bar{\boldsymbol{\sigma}}}_{n+\frac{1}{2}}\approx\frac{\bar{\boldsymbol{\sigma}}_{n+1}-\bar{\boldsymbol{\sigma}}_{n}}{\Delta t}.
\end{equation}
From Appendix A we know that the Lie-derivative $\acctriang{\boldsymbol{\sigma}}$ and the derivative in the undeformed configuration $\dot{\bar{\boldsymbol{\sigma}}}$ relate to each other via:
\begin{equation}
 \dot{\bar{\boldsymbol{\sigma}}}=\boldsymbol{F}^{-1}\cdot\acctriang{\boldsymbol{\sigma}}\cdot\boldsymbol{F}^{-T}.
\end{equation}
Let us apply this transformation to both sides of the constitutive Equation of \ref{AppB_consteq}:
\begin{equation}
\boldsymbol{F}^{-1}\cdot\acctriang{\boldsymbol{\sigma}}\cdot\boldsymbol{F}^{-T} = \frac{1}{J}\; \boldsymbol{F}^{-1}\cdot(\boldsymbol{\mathbb L}:\boldsymbol{D})\cdot\boldsymbol{F}^{-T}.
\end{equation} 

Substituting the relation between $\boldsymbol{D}$ and $\dot{\boldsymbol{E}}$ given by:
\begin{equation}
\boldsymbol{D} = \boldsymbol{F}^{-T}\cdot\dot{\boldsymbol{E}}\cdot\boldsymbol{F}^-1,
\end{equation}
leads to:
\begin{equation}
\boldsymbol{F}^{-1}\cdot\acctriang{\boldsymbol{\sigma}}\cdot\boldsymbol{F}^{-T} = \frac{1}{J}\; \boldsymbol{F}^{-1}\cdot\Big(\boldsymbol{\mathbb L}:\big(\boldsymbol{F}^{-T}\cdot\dot{\boldsymbol{E}}\cdot\boldsymbol{F}^{-1}\big)\Big)\cdot\boldsymbol{F}^{-T}.
\end{equation} 
Also, a part of the term on the right hand side can be rewritten to match the expression for the spatial elasticity tensor $\boldsymbol{\mathbb L}$ that was given in Equation \ref{Ch3_CL}. 
\begin{equation}
\frac{1}{J}\; \boldsymbol{F}^{-1}\cdot\Big(\boldsymbol{\mathbb L}:\big(\boldsymbol{F}^{-T}\cdot\dot{\boldsymbol{E}}\cdot\boldsymbol{F}^{-1}\big)\Big)\cdot\boldsymbol{F}^{-T}= \frac{1}{J}\; (\boldsymbol{F}^{-1}\overline{\otimes}\boldsymbol{F}^{-1}):\boldsymbol{\mathbb L}:(\boldsymbol{F}^{-T}\underline{\otimes}\boldsymbol{F}^{-T}):\dot{\boldsymbol{E}}.
\end{equation}










ddddddd
\vspace{4cm}
\noindent\makebox[\linewidth]{\rule{\paperwidth}{0.4pt}}
\begin{equation}
\boldsymbol{F}^{-1}\cdot\acctriang{\boldsymbol{\sigma}}\cdot\boldsymbol{F}^{-T} = \frac{1}{J}\; \boldsymbol{\mathbb L}:(\boldsymbol{D}-\boldsymbol{D}^p).
\end{equation} 

\begin{equation}
(\boldsymbol{F}^{-1}\overline{\otimes}\boldsymbol{F}^{-1}):\boldsymbol{\mathbb L}:(\boldsymbol{F}^{-T}\underline{\otimes}\boldsymbol{F}^{-T})
\end{equation}
The first step of the pull-back push-forward procedure tells us to transform the stress back to a state which is insensitive for rigid body transformations. Since the Lie-derivative is used, this state is the the undeformed (reference) configuration. We perform the transformation to the undeformed configuration by using the deformation gradients $\boldsymbol{F}_n$ and $\boldsymbol{F}_{n+1}$. In the remainder of this appendix, a bar over a symbol denotes that it has been transformed to the undeformed configuration. The Cauchy stresses in the reference configuration become:
\begin{equation}
\left.\begin{aligned}
\bar{\boldsymbol{\sigma}}_{n+1} &=\boldsymbol{F}_{n+1}^{-1}\cdot\boldsymbol{\sigma}_{n+1}\cdot\boldsymbol{F}_{n+1}^{-T}\\
\bar{\boldsymbol{\sigma}}_{n} &= \boldsymbol{F}^{-1}_{n}\cdot\boldsymbol{\sigma}_{n}\cdot\boldsymbol{F}_{n}^{-T}.
\end{aligned}\right.
\label{Ch3_stressbar}
\end{equation}

The second step of the pull-back push-forward procedure describes that the time derivative of the stress tensor must be calculated in the transformed state.  Based on the work of De Souza Neto et al.\cite{compmethodsplasticity}{\color{red}{check this reference}}, the mid-point rule is used to approximate this time derivative. This means that a time $t_{n+\frac{1}{2}}$ is defined, which is in the middle of the defined time interval.\\
In this mid-point, we define a deformation gradient tensor $\boldsymbol{F}_{n+\frac{1}{2}}$, calculated through a simple interpolation:
\begin{equation}
\boldsymbol{F}_{n+\frac{1}{2}} = \frac{1}{2}(\boldsymbol{F}_{n}+\boldsymbol{F}_{n+1})
\end{equation}
At the mid-point, also a rate of deformation tensor can be calculated, given by:
\begin{equation}
\boldsymbol{D}_{n+\frac{1}{2}} = \frac{1}{2}(\boldsymbol{L}_{n+\frac{1}{2}}+\boldsymbol{L}_{n+\frac{1}{2}}^T)\quad\text{with}\quad \boldsymbol{L}_{n+\frac{1}{2}} = \frac{\boldsymbol{F}_{n+1}-\boldsymbol{F}_n}{\Delta t}\cdot \boldsymbol{F}_{n+\frac{1}{2}}^{-1}
\end{equation}

The value of $\dot{\bar{\boldsymbol{\sigma}}}$ in the middle of the time interval can be approximated by:
{\begin{equation}
\dot{\bar{\boldsymbol{\sigma}}}_{n+\frac{1}{2}}\approx\frac{\bar{\boldsymbol{\sigma}}_{n+1}-\bar{\boldsymbol{\sigma}}_{n}}{\Delta t}.
\end{equation}

We know that, by definition, the transformation of the Lie-derivative of the Cauchy stress tensor back to the undeformed configuration will yield $\dot{\bar{\boldsymbol{\sigma}}}$:
\begin{equation}
 \dot{\bar{\boldsymbol{\sigma}}}=\boldsymbol{F}^{-1}\cdot\acctriang{\boldsymbol{\sigma}}\cdot\boldsymbol{F}^{-T}.
\end{equation}
If we apply this transformation to both sides of the constitutive law of Equation \ref{AppB_consteq}, we find that:
\begin{equation}
\boldsymbol{F}^{-1}\cdot\acctriang{\boldsymbol{\sigma}}\cdot\boldsymbol{F}^{-T} = \frac{1}{J}\boldsymbol{\mathbb C}:\dot{\boldsymbol{E}}
\end{equation}



The time derivative of the transformed Cauchy stress tensor, $\dot{\bar{\boldsymbol{\sigma}}}$, can be approximated in this mid-point by:
\begin{equation}
    \dot{\bar{\boldsymbol{\sigma}}}\approx\frac{\bar{\boldsymbol{\sigma}}_{n+1}-\bar{\boldsymbol{\sigma}}_{n}}{\Delta t} \approx \frac{1}{J_{n+\frac{1}{2}}}\; \bar{\boldsymbol{\mathbb{L}}}:(\bar{\boldsymbol{D}}_{n+\frac{1}{2}}-\bar{\boldsymbol{D}}^p_{n+\frac{1}{2}}),
    \label{Ch3_sigmabardot}
\end{equation}
with the spatial elasticity tensor $\boldsymbol{\mathbb{L}}$ and the rate of deformation tensors $\boldsymbol{D}$ and $\boldsymbol{D}^p$ evaluated at time $t_{n+\frac{1}{2}}$ and mapped back to the reference configuration by:
\begin{equation}
\left.\begin{aligned}
\bar{\boldsymbol{D}}_{n+\frac{1}{2}} =& \;\boldsymbol{F}^{-1}_{n+\frac{1}{2}} \cdot{\boldsymbol{D}}_{n+\frac{1}{2}}\cdot\boldsymbol{F}^{-T}_{n+\frac{1}{2}} \\
\bar{\boldsymbol{D}}^p_{n+\frac{1}{2}} =& \;\boldsymbol{F}^{-1}_{n+\frac{1}{2}} \cdot{\boldsymbol{D}}^p_{n+\frac{1}{2}}\cdot\boldsymbol{F}^{-T}_{n+\frac{1}{2}} \\
\bar{\boldsymbol{\mathbb{L}}} =& \;(\boldsymbol{F}^{-1}_{n+\frac{1}{2}}\overline{\otimes}\boldsymbol{F}^{-1}_{n+\frac{1}{2}}):\boldsymbol{\mathbb L}:(\boldsymbol{F}^{-T}_{n+\frac{1}{2}}\underline{\otimes}\boldsymbol{F}^{-T}_{n+\frac{1}{2}}) = \boldsymbol{\mathbb C}.    
\end{aligned}\right.
\label{AppB:incrementprop}
\end{equation}
From this last transformation, the inverse relation of Equation \ref{Ch3_CL} can be recognized. Thus, we can simply use the material elasticity tensor $\boldsymbol{\mathbb C}$ in stead of the spatial elasticity tensor $\bar{\boldsymbol{\mathbb{L}}}$ in the expression for $\dot{\bar{\boldsymbol{\sigma}}}$.\\
\newline
The third and last step of the pull-back push-forward procedure is to transform the time derivative to the current configuration, which is $t_{n+1}$. This transformation is done by pre- and post-multiplying both sides with $\boldsymbol{F}_{n+1}$ and $\boldsymbol{F}^T_{n+1}$ respectively.\\
Performing this transformation, substituting $\boldsymbol{\mathbb C}_{n+\frac{1}{2}}$ for $\bar{\boldsymbol{\mathbb L}}_{n+\frac{1}{2}}$ and rearranging leads to an expression for the new Cauchy stress defined in the current configuration, given by:
\begin{equation}
    {\boldsymbol{\sigma}}_{n+1}\approx\boldsymbol{F}_{n+1}\cdot\Big[\bar{\boldsymbol{\sigma}}_{n}+ \frac{\Delta t}{J_{n+\frac{1}{2}}}\; {\boldsymbol{\mathbb{C}}}:(\bar{\boldsymbol{D}}_{n+\frac{1}{2}}-\bar{\boldsymbol{D}}^p_{n+\frac{1}{2}})\Big]\cdot\boldsymbol{F}^T_{n+1}.
    \label{Ch3_sigmabarnew}
\end{equation}
